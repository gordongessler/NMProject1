
\author{
        Maciej Sobczynski\\
}
\date{\today}

\title{\vspace{-2.0cm}Task 9\\Compute the $LL^T$ factorization of $A$, where $A\in{\rm I\!R}^{nxn}$ is a tridiagonal and symmetric positive definite. Use this decomposition to solve a system of equations $A^TAx=b$}
\documentclass[12pt]{article}
\usepackage{listings}  
\providecommand{\e}[1]{\ensuremath{\times 10^{#1}}}
\begin{document}
\lstset{language=Matlab}
\maketitle

\section{Method description}

In this task, we consider only matrices that are tridiagonal and symmetric positive definite. This means that the matrices have non-zero elements only on the main diagonal and the first diagonals below and above the main one. Because the matrices considered are symmetric values on both diagonals next to the main diagonal have to be equal. Therefore the matrices considered in this task take the following form

$$
A= \left[
\begin{array}{ccccc}
\alpha_1 & \beta_1 & 0 &\cdots&0\\
\beta_1 & \alpha_2 & \beta_2 & &\vdots\\
0 &\ddots&\ddots&\ddots& 0\\
\vdots&  & \beta_{n-2} & \alpha_{n-1} & \beta_{n-1}\\
0 & \cdots& 0 & \beta_{n-1} & \alpha_n
\end{array}
\right]
$$

The task is to perform $LL^T$ (Cholesky) factorization on the matrix A and use the factorization to solve a system of equations $A^TAx=b$ where 

$$
x=\left(
\begin{array}{c}
x_1\\
x_2\\
\vdots\\
x_n\\
\end{array}
\right)
\ \
b=\left(
\begin{array}{c}
b_1\\
b_2\\
\vdots\\
b_n\\
\end{array}
\right)
$$

The factorization is obtained in the following way\\

Let $D$ be $n$x$n$ diagonal matrix with diagonal elements $\delta_j$ for $j = 1,...,n$ and
$$
L= \left[
\begin{array}{ccccc}
1 &  &  & & \\
l_1 & 1 &   & & \\
  &\ddots&\ddots& &  \\
 &  & l_{n-2} & 1 &  \\
  &  &   & l_{n-1} & 1
\end{array}
\right]
$$
$$
D= \left[
\begin{array}{ccccc}
\delta_1 &  &  & & \\
 & \delta_2 &   & & \\
  & &\ddots& &  \\
 &  &  & \delta_{n-1} &  \\
  &  &   &  & \delta_n
\end{array}
\right]
$$
$$\delta_1=\alpha_1, \ l_1=\frac{\beta_1}{\delta_1}$$
$$\delta=\alpha_j-\frac{\beta_{j-1}^2}{\delta_{j-1}},\ j=2,...,n\ ,\ l_j=\frac{\beta_j}{\delta_j},\ j=2,...,n-1$$

The factorization cannot be completed if any diagonal element $\delta$ is zero. This situation does not occur when A is positive definite.\\

The factorization can be obtained from $D$ in the following fashion
$$
A = L^C(L^C)^T
$$
Where $L^C=L+D^{\frac{1}{2}}$
$$
L^C = \left[
\begin{array}{ccccc}
\sqrt{\delta_1} &  &  & & \\
\frac{\beta_1}{\sqrt{\delta_1}} & \sqrt{\delta_2} &   & & \\
  &\ddots&\ddots& &  \\
 &  & \frac{\beta_{n-2}}{\sqrt{\delta_{n-2}}} & \sqrt{\delta_{n-1}} &  \\
  &  &   & \frac{\beta_{n-1}}{\sqrt{\delta_{n-1}}} & \sqrt{\delta_n}
\end{array}
\right]
$$

Once the decomposition is obtained, the system of equations can be solved by solving $L^Cy=b$ for $y$ by forward substitution, and $(L^C)^Tx=y$ for $x$ by back substitution to get the final result.                  


\section{Program description}
The program consists of following files,
\begin{enumerate}
\item \textit{LLT.m} - The main function of the program. Can be run with the \textit{LLT(A,b)} command. The function performs $LL^T$ factorization with the help of \textit{cholesky.m} and uses the result of the factorization to solve the system of equations. The function returns the vector of solutions to the system.
\item \textit{cholesky.m} - A function that performs $LL^T$ decomposition on a tridiagonal and symmetric positive definite matrix. The code is optimised to only calculate factorization for matrices of this type and is not suitable for other matrices.
\item \textit{examples.m} - A simple script to perform the calculations for provided examples. It calls the \textit{LLT} function on predefined matrices and vectors
\item \textit{ForwardSub.m} - A simple function to solve a system of linear equations $Ax=b$  where $A$ is lower triangular by forward substitution
\item \textit{BackwardSub.m} - A simple function to solve a system of linear equations $Ax=b$
where $A$ is upper triangular by using backward substitution.
\item \textit{tridiag.m} - A modified version of the code provided during the laboratories. Creates a tridiagonal matrix where diagonals above and below the main one are identical.
\end{enumerate}
The \textit{cholesky.m} and \textit{LLT.m} functions will stop calculations and display an error message if the provided matrix or vector does not fit the conditions. That is, when the vector is not a column vector or it's height is lower than the height of the matrix, when the matrix is not square, tridiagonal, symmetric or positive definite.

\section{Numerical examples}
\subsection{Example 1}
$$A = \left(\begin{array}{ccccc} 3 & 1 & 0 & 0 & 0\\ 1 & 3 & 1 & 0 & 0\\ 0 & 1 & 3 & 1 & 0\\ 0 & 0 & 1 & 3 & 1\\ 0 & 0 & 0 & 1 & 3 \end{array}\right)\ 
b = \left(\begin{array}{c} 1\\ 1\\ 1\\ 1\\ 1 \end{array}\right)$$
$$L = \left(\begin{array}{ccccc} \sqrt{3} & 0 & 0 & 0 & 0\\ \frac{\sqrt{3}}{3} & \frac{2\,\sqrt{2}\,\sqrt{3}}{3} & 0 & 0 & 0\\ 0 & \frac{\sqrt{2}\,\sqrt{3}}{4} & \frac{\sqrt{2}\,\sqrt{21}}{4} & 0 & 0\\ 0 & 0 & \frac{2\,\sqrt{2}\,\sqrt{21}}{21} & \frac{\sqrt{21}\,\sqrt{55}}{21} & 0\\ 0 & 0 & 0 & \frac{\sqrt{21}\,\sqrt{55}}{55} & \frac{12\,\sqrt{55}}{55} \end{array}\right) \
x=\left(\begin{array}{c} \frac{5}{18}\\ \frac{1}{6}\\ \frac{2}{9}\\ \frac{1}{6}\\ \frac{5}{18} \end{array}\right)$$

Error: $9.3847\e{-17}$ Condition number: $3.7321$

\subsection{Example 2}
$$A = \left(\begin{array}{cccccccccc} 2 & 1 & 0 & 0 & 0 & 0 & 0 & 0 & 0 & 0\\ 1 & 2 & 1 & 0 & 0 & 0 & 0 & 0 & 0 & 0\\ 0 & 1 & 2 & 1 & 0 & 0 & 0 & 0 & 0 & 0\\ 0 & 0 & 1 & 2 & 1 & 0 & 0 & 0 & 0 & 0\\ 0 & 0 & 0 & 1 & 2 & 1 & 0 & 0 & 0 & 0\\ 0 & 0 & 0 & 0 & 1 & 2 & 1 & 0 & 0 & 0\\ 0 & 0 & 0 & 0 & 0 & 1 & 2 & 1 & 0 & 0\\ 0 & 0 & 0 & 0 & 0 & 0 & 1 & 2 & 1 & 0\\ 0 & 0 & 0 & 0 & 0 & 0 & 0 & 1 & 2 & 1\\ 0 & 0 & 0 & 0 & 0 & 0 & 0 & 0 & 1 & 2 \end{array}\right)\ 
b = \left(\begin{array}{c} 3\\ 4\\ 4\\ 4\\ 4\\ 4\\ 4\\ 4\\ 4\\ 3 \end{array}\right)$$
$$L=\left(\begin{array}{cccccccccc} \sqrt{2} & 0 & 0 & 0 & 0 & 0 & 0 & 0 & 0 & 0\\ \frac{\sqrt{2}}{2} & \frac{\sqrt{2}\,\sqrt{3}}{2} & 0 & 0 & 0 & 0 & 0 & 0 & 0 & 0\\ 0 & \frac{\sqrt{2}\,\sqrt{3}}{3} & \frac{2\,\sqrt{3}}{3} & 0 & 0 & 0 & 0 & 0 & 0 & 0\\ 0 & 0 & \frac{\sqrt{3}}{2} & \frac{\sqrt{5}}{2} & 0 & 0 & 0 & 0 & 0 & 0\\ 0 & 0 & 0 & \frac{2\,\sqrt{5}}{5} & \frac{\sqrt{5}\,\sqrt{6}}{5} & 0 & 0 & 0 & 0 & 0\\ 0 & 0 & 0 & 0 & \frac{\sqrt{5}\,\sqrt{6}}{6} & \frac{\sqrt{6}\,\sqrt{7}}{6} & 0 & 0 & 0 & 0\\ 0 & 0 & 0 & 0 & 0 & \frac{\sqrt{6}\,\sqrt{7}}{7} & \frac{2\,\sqrt{2}\,\sqrt{7}}{7} & 0 & 0 & 0\\ 0 & 0 & 0 & 0 & 0 & 0 & \frac{\sqrt{2}\,\sqrt{7}}{4} & \frac{3\,\sqrt{2}}{4} & 0 & 0\\ 0 & 0 & 0 & 0 & 0 & 0 & 0 & \frac{2\,\sqrt{2}}{3} & \frac{\sqrt{10}}{3} & 0\\ 0 & 0 & 0 & 0 & 0 & 0 & 0 & 0 & \frac{3\,\sqrt{10}}{10} & \frac{\sqrt{10}\,\sqrt{11}}{10} \end{array}\right)\
x=\left(\begin{array}{c} 1\\ 1\\ 1\\ 1\\ 1\\ 1\\ 1\\ 1\\ 1\\ 1 \end{array}\right)
$$

Error: $1.1332\e{-16}$ Condition number: $48.3742$

\subsection{Example 3}
$$A=\left(\begin{array}{cccccc} 5 & 1 & 0 & 0 & 0 & 0\\ 1 & 5 & 2 & 0 & 0 & 0\\ 0 & 2 & 5 & 3 & 0 & 0\\ 0 & 0 & 3 & 5 & 2 & 0\\ 0 & 0 & 0 & 2 & 5 & 1\\ 0 & 0 & 0 & 0 & 1 & 5 \end{array}\right)\
b=\left(\begin{array}{c} 1\\ 3\\ 4\\ 5\\ 2\\ 0 \end{array}\right)
$$
$$
L=\left(\begin{array}{cccccc} \sqrt{5} & 0 & 0 & 0 & 0 & 0\\ \frac{\sqrt{5}}{5} & \frac{2\,\sqrt{5}\,\sqrt{6}}{5} & 0 & 0 & 0 & 0\\ 0 & \frac{\sqrt{5}\,\sqrt{6}}{6} & \frac{5\,\sqrt{6}}{6} & 0 & 0 & 0\\ 0 & 0 & \frac{3\,\sqrt{6}}{5} & \frac{\sqrt{71}}{5} & 0 & 0\\ 0 & 0 & 0 & \frac{10\,\sqrt{71}}{71} & \frac{\sqrt{71}\,\sqrt{255}}{71} & 0\\ 0 & 0 & 0 & 0 & \frac{\sqrt{71}\,\sqrt{255}}{255} & \frac{2\,\sqrt{255}\,\sqrt{301}}{255} \end{array}\right)\
x=\left(\begin{array}{c} \frac{23}{301}\\ \frac{186}{301}\\ -\frac{25}{301}\\ \frac{319}{301}\\ -\frac{15}{602}\\ \frac{3}{602} \end{array}\right)
$$
Error: $9.8126\e{-17}$ Condition number: $9.5416$
\subsection{Example 4}
A is a random nxn matrix
\section{Analysis of results}
We worked hard, and achieved very little.

\section{Source code}
\subsection{LTT.m}
\lstinputlisting{LLT.m}

\subsection{cholesky.m}
\lstinputlisting{cholesky.m}

\subsection{examples.m}
\lstinputlisting{examples.m}

\subsection{ForwardSub.m}
\lstinputlisting{ForwardSub.m}

\subsection{BackwardSub.m}
\lstinputlisting{BackwardSub.m}

\subsection{tridiag1.m}
\lstinputlisting{tridiag1.m}
\bibliography{main}

\end{document}
This is never printed
  